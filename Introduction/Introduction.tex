\section{Introduction}
\label{sec:introduction}
For the last two decades smartphones have been on the rise and caused many things to change in software development.
Forster Research predicted one billion smartphone consumers in 2016 (\cite{schadler2012mobile}).
It actually were 3.668 billion and in 2021 it were already 6.259 billion\footnote{https://www.statista.com/statistics/330695/number-of-smartphone-users-worldwide/ (last visited: 25.06.22, 22:26)}.
With the rapid increase in smartphone usage, approaches like mobile-first came up where applications get developed for mobile devices first and other devices like desktop computers second (\cite{wroblewski2012mobile}).

This thesis however will implement a mobile version of \gls{see}, which already has an adaption for desktop devices.
The \gls{see} project aims to connect people regardless of space in a virtual room to analyze code structure.  
In the current state of the \gls{see} project the virtual room can be accessed with \gls{vr} glasses or a desktop device.

The motivation of this master thesis shall be to integrate mobile devices into the \gls{see} project. 
Therefore, solutions need to be found to transfer the given operations to mobile devices considering the constraints and benefits of rather small devices.
A concept shall be found that integrates the given operations in an innovative way. 
The main challenges will be finding solutions to the constraints of having a smaller screen and the control operations, that use different input methods than the other systems. 
After implementing the concept as a prototype, the new ways of operating in the virtual room shall be evaluated in comparison with the desktop version. 
In conclusion this work shall give insight whether the mobile support is a valuable extension to the system or not. 
In addition to that, it shall show an approach of how to convert a rather complex system to mobile devices.

\subsection{Research Question}
\label{research}
The central research question of this thesis is: 
\textit{Are Android smartphones suitable of working with \gls{see}?}
To answer this question, part of this thesis will be to implement an \gls{android} version of \gls{see} and evaluate it in a user study.
The user study will compare the desktop version with the mobile version of \gls{see}.


\subsection{Thesis Structure}
This thesis will begin with discussing the fundamentals of this work in chapter \ref{sec:fundamentals}.
This will include the \gls{see} project itself with its use of \glspl{city} as well as a short introduction to \gls{unity} and its key concepts needed for the implementation.

Afterward, the concept for the mobile version will be explained in chapter \ref{section:concept}.
Part of this chapter will be the planned interface and interactions for the implementation as well as a list of requirements that need to be fulfilled for a working prototype.

The concept will be followed by an implementation, which will be unfolded in chapter \ref{section:implementation}.
That chapter will discuss how certain parts are implemented and why they are implemented in that way.

With the given implementation from chapter \ref{section:implementation}, the earlier announced evaluation will be executed in chapter \ref{section:evaluation}.
Therefore, 20 participants took part in a user study and their feedback will be the foundation for the answer of the research question.

Finally, a conclusion of this thesis as well as further ideas for future projects will be given in chapter \ref{section:conclusion}.