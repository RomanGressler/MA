% Template by Falko Galperin <falko1@tzi.de>, 2022, V1.0
% Main styling of the template comes from the ClassicThesis package by André Miede.

% IMPORTANT POINTS BELOW:
% - Look at the options between "CONFIGURATION HERE" and "CONFIGURATION ENDS".
% - Abstract.tex contains the abstract (can be disabled).
% - Inhalt.tex contains the contents of your thesis.
% - Glossar.tex contains the glossary (can be disabled) and further options.

% Less important points:
% - You can compile a PDF using `make` and clean up unnecessary files 
%   using `make clean`. `make pdf` combines both.
% - CMYK color space will be used so that printers don't get confused.
% - Feel free to configure other parts outside of the section below too,
%   e.g. uncomment the list of tables if you want that.

%%% CONFIGURATION HERE %%%

% Enter information about your thesis here, adjust as necessary.
% I recommend keeping the `\xspace` at the end.
\newcommand{\myTitle}{System porting to mobile devices at the example of the SEE project\xspace}
\newcommand{\mySubtitle}{Master Thesis\xspace}
\newcommand{\myName}{Roman Gressler\xspace}
\newcommand{\myNumber}{3217822\xspace}
\newcommand{\myProf}{Prof.\ Dr.\ Rainer Koschke\xspace}
\newcommand{\myOtherProf}{Prof.\ Dr.\ Zwetachter\xspace}
\newcommand{\myDepartment}{Faculty 3 --- Mathematics and Computer Science\xspace}
\newcommand{\myDegree}{Computer Science\xspace}
\newcommand{\myDate}{\today}
\newcommand{\myVersion}{\classicthesis}

% Whether links should be colored.
\def\colorLinks{true}

% Color for citations.
\def\citeColor{Periwinkle}

% Color for URLs.
\def\urlColor{Cyan}

% The file in which your BibLaTeX sources reside.
% Remove this line to disable the bibliography.
\def\sourcesFile{sources.bib}

% If you won't use the glossary template, remove the following line.
% Otherwise, set this to the filename of the glossary LaTeX file.
\def\glossaryFile{Glossar.tex}

% If you don't want to have an abstract, remove the following line.
% Otherwise, set this to the filename of the LaTeX file containing the abstract.
\def\abstractFile{Abstract.tex}

% Despite the seemingly broad implications of this option, it'll simply include a 
% disclaimer for readers in case you don't want to use something 
% like the "Gendersternchen". Set it to 1 to enable this disclaimer.
\def\disableGender{0}

% Set this to 1 if you want to use \section for the highest level of headings.
% Otherwise, \chapter will be used.
% IMPORTANT NOTE: This template assumes this value will be 1.
% If you choose to set it to something other than 1, you'll need to
% adjust this template's usage of \section and change it to \chapter.
\def\disableChapter{1}

% These are ClassicThesis options. 
% Don't forget to set drafting=false for the final version.
% If you have pdflatex, I recommend eulermath=true, it looks a bit better in my opinion.
\PassOptionsToPackage{
  drafting=true,  % print version information on the bottom of the pages
  tocaligned=false, % the left column of the toc will be aligned (no indentation)
  dottedtoc=true,  % page numbers in ToC flushed right
  eulerchapternumbers=true, % use AMS Euler for chapter font (otherwise Palatino)
  linedheaders=false,  % chaper headers will have line above and beneath
  floatperchapter=true,  % numbering per chapter for all floats (i.e., Figure 1.1)
  eulermath=false,  % use Euler fonts for mathematical formulae (only with pdfLaTeX)
  beramono=true,    % toggle a different monospaced font (w/ bold)
  style=classicthesis % classicthesis, arsclassica
}{classicthesis}

%%% CONFIGURATION ENDS %%%
% (Of course, feel free to modify the rest of the below code too.)

% ClassicThesis causes these false positive warnings, hence we silence them.
\RequirePackage{silence} % :-\
    \WarningFilter{scrreprt}{Usage of package `titlesec'}
    \WarningFilter{titlesec}{Non standard sectioning command}

\documentclass[twoside,openright,titlepage,numbers=noenddot,%headlines,
               headinclude,footinclude,cleardoublepage=empty,abstract=on,
               BCOR=5mm,paper=a4,listof=totocnumbered]{scrreprt}
\usepackage[utf8]{inputenc}
\usepackage[T1]{fontenc}

\PassOptionsToPackage{hyphens}{url}

\usepackage[english]{babel}
\usepackage[dvipsnames,usenames,cmyk]{xcolor}
\usepackage{hyperref} 
\usepackage{xspace}
% \usepackage[style=alphabetic]{biblatex}
\usepackage[round]{natbib} % Literatur und Referenzen
\usepackage{bm}
\usepackage{dsfont}
\usepackage{sidenotes}
\usepackage{newpxtext}
\usepackage{epigraph}
\usepackage{etoolbox}
\usepackage{multicol}
\usepackage{graphicx}
\usepackage{wrapfig}
\usepackage{microtype}
% Note: If you enable this, you need the -shell-escape flag.
%\usepackage{minted}
%\usemintedstyle{friendly}
\usepackage{xifthen}
\usepackage{enumitem}
\usepackage{calc}
\usepackage{subfig}
\usepackage{amssymb}
\usepackage{scalerel}
\usepackage[noend]{algpseudocode}
\usepackage{algorithm}
\usepackage{xstring}

\usepackage[normalem]{ulem}
\setlength{\columnsep}{0,1cm}

\usepackage{classicthesis}
\usepackage{titlesec}
\usepackage{multirow}
\usepackage{graphicx}
% Make headings a little more discernable.
\titleformat*{\chapter}{\LARGE\bfseries}
\titleformat*{\section}{\Large\bfseries}
\titleformat*{\subsection}{\large\bfseries}

\hypersetup{colorlinks=\colorLinks, citecolor=\citeColor, 
    urlcolor=\urlColor, linktocpage=true, bookmarksnumbered,
  pdftitle={\myTitle},%
  pdfauthor={\textcopyright\ \myName, University Bremen},%
  pdfsubject={\mySubtitle},%
  pdfcreator={pdfLaTeX},%
  pdfproducer={LaTeX with classicthesis and FG-V1.0}%
}

%\ifdef{\sourcesFile}{\addbibresource{\sourcesFile}}{}

% Allow " instead of "` and "'
\usepackage{csquotes}
\MakeOuterQuote{"}

% Uncomment this if you want textsc to be a little (x 1.15) bigger.
%\usepackage{letltxmacro,scalefnt}
%\newcommand{\bigtextsc}[1]{\textsc{\scalefont{1.15}#1}}

% Some custom commands:
% Proper spacing for "z.B."
\newcommand{\zB}{z.\,B.\xspace}
% Command for TODOs, use either \TODO or \TODO{Details}
\newcommand{\TODO}[1]{\textbf{\textcolor{red}{\ifthenelse{\isempty{#1}}{TODO!}{TODO: #1}}}\xspace}
% Properly color Axivion's name.
\newcommand{\Axivion}{\textsc{{\color{black}{A}}{\color{red}{x}}{\color{black}{ivion}}}}

\newcommand{\Regie}[1]{\textbf{Regie:} #1}
\newcommand{\Stil}[1]{\textbf{Stil:} #1}

\newtheorem{file}{Related file}

% Improve caption fonts.
\setkomafont{caption}{\footnotesize\itshape}
\setkomafont{captionlabel}{\usekomafont{caption}}

% Import glossary if necessary.
\ifdef{\glossaryFile}{\input{\glossaryFile}}{}

\begin{document}

\title{\myTitle}
\subtitle{\mySubtitle}
\author{\myName}
\date{\myDate} 

\raggedbottom
\selectlanguage{english}
\captionsetup[subfigure]{justification=centering}

\pagestyle{plain}
\pagenumbering{roman}

\begin{titlepage}
    \begin{addmargin}[-1cm]{-3cm}
	\begin{center}
		\Huge
		\vspace*{1cm}
        \begingroup
            \myTitle \\ \bigskip
        \endgroup
		\LARGE
        \mySubtitle\\
		\vspace{3cm}
 		\Large
		\myName\\
		\vspace{6pt}
        Matriculation number: \myNumber\\
		\vspace{1cm}
		\myDate\\
		\vspace{2cm}
 		\includegraphics[width=6cm]{unibremen}\\
 		\vspace{1cm}
 		\large
 		\myDepartment\\
		\myDegree\\
		\vspace{4cm}
		\large
		1. Supervisor: \myProf\\
		2. Supervisor: \myOtherProf\\
		\vspace{1.5cm}
	\end{center}
    \end{addmargin}
\end{titlepage}
\cleardoublepage{}

\ifdef{\abstractFile}{%
\pdfbookmark[1]{Zusammenfassung}{abstract}
\chapter*{Abstract}
\input{\abstractFile}
\cleardoublepage{}
}{}

\pdfbookmark[1]{Erklärung}{erklaerung}
\chapter*{Erklärung}\label{erklaerung}
Ich versichere, diese Arbeit --- sofern dies nicht explizit anders
gekennzeichnet wurde --- ohne fremde Hilfe angefertigt zu haben.  Ich
habe keine anderen als die angegebenen Quellen und Hilfsmittel
benutzt.  Alle Stellen, die wörtlich oder sinngemäß aus
Veröffentlichungen entnommen sind, sind als solche kenntlich gemacht.

\bigskip

\noindent\textit{Bremen, den \today}
\smallskip
\begin{flushright}
    \begin{tabular}{m{5cm}}
        \\ \hline
        \centering\myName \\
    \end{tabular}
\end{flushright}

\vfill

\cleardoublepage{}

\pdfbookmark[1]{Danksagung}{Danksagung}
\begingroup
\let\clearpage\relax
\let\cleardoublepage\relax
\let\cleardoublepage\relax
\chapter*{Acknowledgement}
I want to thank me and only me for putting all this hard work in.


\ifdef{\disableGender}{%
\if\disableGender1
\vfill
\chapter*{Gender}

\vfill
\fi
}{}
\endgroup

\cleardoublepage

\pagestyle{scrheadings}

% Table of contents:
\pdfbookmark[1]{\contentsname}{tableofcontents}
\tableofcontents

\clearpage

\ifdef{\disableChapter}{
\if\disableChapter1
\let\subsubsection\subsection
\let\subsection\section
\let\section\chapter
\fi}{}

\cleardoublepage
\pagenumbering{arabic}

\section{Introduction}
\label{sec:introduction}
For the last two decades smartphones have been on the rise and caused many things to change in software development.
Forster Research predicted one billion smartphone consumers in 2016 (\cite{schadler2012mobile}).
It actually were 3.668 billion and in 2021 it were already 6.259 billion\footnote{https://www.statista.com/statistics/330695/number-of-smartphone-users-worldwide/ (last visited: 25.06.22, 22:26)}.
With the rapid increase in smartphone usage, approaches like mobile-first came up where applications get developed for mobile devices first and other devices like desktop computers second (\cite{wroblewski2012mobile}).

This thesis however will implement a mobile version of \gls{see}, which already has an adaption for desktop devices.
The \gls{see} project aims to connect people regardless of space in a virtual room to analyze code structure.  
In the current state of the \gls{see} project the virtual room can be accessed with \gls{vr} glasses or a desktop device.

The motivation of this master thesis shall be to integrate mobile devices into the \gls{see} project. 
Therefore, solutions need to be found to transfer the given operations to mobile devices considering the constraints and benefits of rather small devices.
A concept shall be found that integrates the given operations in an innovative way. 
The main challenges will be finding solutions to the constraints of having a smaller screen and the control operations, that use different input methods than the other systems. 
After implementing the concept as a prototype, the new ways of operating in the virtual room shall be evaluated in comparison with the desktop version. 
In conclusion this work shall give insight whether the mobile support is a valuable extension to the system or not. 
In addition to that, it shall show an approach of how to convert a rather complex system to mobile devices.

\subsection{Research Question}
\label{research}
The central research question of this thesis is: 
\textit{Are Android smartphones suitable of working with \gls{see}?}
To answer this question, part of this thesis will be to implement an \gls{android} version of \gls{see} and evaluate it in a user study.
The user study will compare the desktop version with the mobile version of \gls{see}.


\subsection{Thesis Structure}
This thesis will begin with discussing the fundamentals of this work in chapter \ref{sec:fundamentals}.
This will include the \gls{see} project itself with its use of \glspl{city} as well as a short introduction to \gls{unity} and its key concepts needed for the implementation.

Afterward, the concept for the mobile version will be explained in chapter \ref{section:concept}.
Part of this chapter will be the planned interface and interactions for the implementation as well as a list of requirements that need to be fulfilled for a working prototype.

The concept will be followed by an implementation, which will be unfolded in chapter \ref{section:implementation}.
That chapter will discuss how certain parts are implemented and why they are implemented in that way.

With the given implementation from chapter \ref{section:implementation}, the earlier announced evaluation will be executed in chapter \ref{section:evaluation}.
Therefore, 20 participants took part in a user study and their feedback will be the foundation for the answer of the research question.

Finally, a conclusion of this thesis as well as further ideas for future projects will be given in chapter \ref{section:conclusion}.
\section{Concept}
\label{section:concept}
In this section a concept of a mobile \gls{see} version will be presented. 
Therefore, a prototype will be created to point out the features that a mobile version of \gls{see} requires.

Prototypes are a common way to express the needs of a system. 
It is a low-cost way of planning an implementation, that can highlight challenges regarding constraints of a system early on.

Even though a prototype will never be able to show every aspect and need of a complex system, it should still help to answering questions like: 
How should the system feel? How should it be implemented, and what are the key features? (\cite{houde1997prototypes}) 

\gls{see} is meant to be used by multiple platforms such as desktop devices, mobile devices and virtual reality devices.
Each device has different interaction constrains. 
While a desktop user will control the player with mouse and keyboard a mobile user will interact with virtual joysticks on a touchscreen.
Selecting nodes of a \gls{city} will be done by clicking it with a mouse on desktop devices, while a mobile device will require a touch input.

\subsection{Interface}
\label{sec:interface}
In the following a paper prototype will be presented that marks out a concept for the mobile interface.
Since the field of mobile development is quite young there few guidelines regarding the design of mobile device interfaces.
A guideline that is widely accepted is problematic to find (\cite{renaud2017demarcating}; \cite{punchoojit2017usability}). 

Major differences to desktop environments are the screen size, forms of input and input feedback.
To assure as much space is used for the actual interaction of the app the menu should just take as much space as needed.
As a study has found out, a size of at least 8*8 mm is needed to reduce error rates selecting the right button (\cite{conradi2015optimal}; \cite{parhi2006target}).

Moving the player will be handled with virtual joysticks as seen in figure \ref{fig:joystick_prototype}.
The left joystick will move the player through the virtual room and the right will move the camera angle or in other word the direction the player looks at.
The joysticks are placed in the left and right corner and should just take as much space as needed to be handled comfortably.
This way the player is able to navigate through the virtual room with his/her thumps while still having enough space to work on the \gls{city}.

\begin{figure}[htb]
    \centering
    \includegraphics[width=1\textwidth]{Concept/img/joystick.png}
    \caption{Joysticks for moving in \gls{see}}\label{fig:joystick_prototype}
\end{figure}

The menu on the top left side seen in figure \ref{fig:quickbar} will be called "quickbar" further on. 
The quickbar can be minimized to safe screen space when not needed. 
The quickbar is designed to offer more general functions that are needed in various situations.
Because there are no shortcuts on mobile devices each function has to have a button to be activated.

The functions are redo and undo which will do an action undone again or revert an action.
Then there is a camera lock that will lock the player's perspective to a certain \gls{city} so that the player can only move around the selected city and move closer or further away from it.
The next function is to rerotate a \gls{city}.
That means the \gls{city} that was last rotated will be set back to its initial state of rotation.
Last but not least there will be a button for recentering the city, which will work quite similar to the rerotate button and center the last moved \gls{city}.
The button on the right can be used to collapse or expand the quickbar.
\begin{figure}[htb]
    \centering
    \includegraphics[width=1\textwidth]{Concept/img/quickbar.png}
    \caption{Quickbar for various interactions in \gls{see}}\label{fig:quickbar}
\end{figure}

On the top right side another menu will be placed that contains different interaction modes.
By clicking a button an interaction mode will be selected and moved to the top right corner.
Also, the menu will be collapsed and only the buttons regarding the selected interaction mode shall be shown.
By clicking the button on the top right again the menu shall expand, and the other interaction modes shall be selectable.
The other buttons shall be kept in the same order to reduce confusion of the user.

The first interaction mode, seen in figure \ref{fig:select}, is for selecting nodes.
Nodes can be selected by being touched and deselected by being touched again.
There can be multiple nodes selected at once.
The hole selection can be deselected by clicking the deselect button next to the select interaction mode button.
Selected nodes shall be highlighted with a different node color and also display their name.

\begin{figure}[htb]
    \centering
    \includegraphics[width=1\textwidth]{Concept/img/menu1.png}
    \caption{Selection mode in \gls{see}}\label{fig:select}
\end{figure}

The second interaction mode, seen in figure \ref{fig:delete}, is for deleting node.
It does not need additional buttons.
Node will be deleted by being touched.-
Unlike in the desktop version there will not be a group deletion interaction because it would require an additional menu panel.
The added functionality would be minimal and selecting a group of nodes, confirming and finally deleting would require a handful more steps and would therefore most likely not be used.

\begin{figure}[htb]
    \centering
    \includegraphics[width=1\textwidth]{Concept/img/menu2.png}
    \caption{Delete mode in \gls{see}}\label{fig:delete}
\end{figure}

The following interaction mode, seen in figure \ref{fig:nodes}, is dedicated to the nodes and edges of a \gls{city}.
Starting on with the "add node" button on the right.
When activated the user can create new node by clicking on a certain spot on the \gls{city} plane. 
The following button on the left is for adding edges.
By selecting two nodes a new edge will be created between them. 
Then, the button one further on the right is for editing nodes.
By touching a node a window will pop up that allows the user to edit the node by changing its name and its type.
Last but not least the button on the left-hand side will be used to scale nodes.
That means the node height and width can be adjusted by first selecting it via touch and then hold a corner and slide it further away from the node center to increase the size or slide it towards the center to decrease the size of the node.
Each button of the node interactions will be marked green after being pressed to indicate that it is active.

\begin{figure}[htb]
    \centering
    \includegraphics[width=1\textwidth]{Concept/img/menu3.png}
    \caption{Node interactions in \gls{see}}\label{fig:nodes}
\end{figure}

Then there will be a button for rotation interactions that can be seen in figure \ref{fig:rotate_proto}.
Starting with the first activatable button that lets the user rotate the hole \gls{city} by touching any point on it and then sliding away from that point.
Similar to that there will be a button that lets the user rotate just a single node on the \gls{city}.
In addition to that there will be a button that activates the so-called "locked-rotation" mode.
While in "locked-rotation" mode the rotation of a node or \gls{city} will be done in eight predefined steps to a full rotation.
Each step will have the same 45° range.
The last button of this group will be for changing the center of the rotations. 
There are to options: the first option is a center of rotation in the middle of the \gls{city} and the second is in the middle of a node selection made with the interactions seen in figure \ref{fig:select}.
The second option can be activated by pressing the last button.

\begin{figure}[htb]
    \centering
    \includegraphics[width=1\textwidth]{Concept/img/menu4.png}
    \caption{Rotation mode in \gls{see}}\label{fig:rotate_proto}
\end{figure}

The last interaction group, seen in figure \ref{fig:move}, is for moving the \gls{city} or a single node.
The move interactions are quite similar to the rotation interactions.
There will be a button to move a hole \gls{city} as well as a button to move only single nodes.
In addition to that there will be a button that restricts the movement of the \gls{city} or node to a predefined direction.
The directions will be again in 45° angles and objects can be moved on a straight line on that angle.
Moving a node or a \gls{city} can be achieved by touching and holding it and then moving it to the desired position.
\begin{figure}[htb]
    \centering
    \includegraphics[width=1\textwidth]{Concept/img/menu5.png}
    \caption{Movement mode in \gls{see}}\label{fig:move}
\end{figure}

\subsection{Interaction}

Smartphones are quite limited in space and there are few input possibilities.
Unlike a desktop computer there is no mouse and there is no physical keyboard.
Smartphones use virtual keyboards but due to the restriction of screen space the keyboard is hidden most of the time.
Which would make keyboard shortcuts uncomfortable because the user has to open the keyboard first.
Therefore, smartphones need different ways of interaction such as touch gestures. 

Zooming in to a \gls{city} happens by scrolling on a desktop environment. 
The is no option to scroll on mobile devices, but there are at least two popular alternatives.
The first option would be to double tap on the \gls{city} to zoom in.
The double tap would zoom in, in predefined steps and after reaching a certain level of closeness it would trigger to zoom out again.
In \gls{see} zooming in, in predefined steps might not be precise enough because there could be a quite large \gls{city} or a rather small one.
Finding predefined steps that would fit every situation is rather hard.
Therefore, a second option by zooming in with a two finger gesture might be better. 
In this option the user uses two fingers and slides them towards each other to zoom in or slides the two fingers away from each other to zoom out.
This way there are no predefined steps necessary and zooming interactions can be done precisely.
\subsection{Requirements}
In the following a list of requirements will be given, which will specify in detail what the implementation of a mobile version has to take care of.
The list will be referred to multiple times in the upcoming realization part in chapter \ref{section:implementation}.
Requirements are essential for the planning phase as they give a good fundamental structure for the developer to rely on (\cite{Robertson2012,Stevens2005}). 
\begin{itemize}
    \item[{[R1]}] The application shall run on \gls{android} devices
    \item[{[R2]}] The application shall be controlled via touchscreen
    \begin{itemize}
        \item [{[R2.1]}] The player and camera shall be moved with virtual joysticks
        \item [{[R2.2]}] Needed shortcuts of the desktop version shall be handled with buttons
        \item [{[R2.3]}] Zooming shall be handled with a two finger gesture
    \end{itemize}
    \item[{[R3]}] The user shall be able to select a node of a \gls{city}
    \begin{itemize}
        \item [{[R3.1]}] After selecting the name of the node shall be shown
        \item [{[R3.2]}] The user shall be able to deselect single nodes or a group of nodes
    \end{itemize}
    \item[{[R4]}] The user shall be able to delete nodes
    \item[{[R5]}] The user shall be able to interact with nodes
    \begin{itemize}
        \item [{[R5.1]}] The user shall be able to add nodes
        \item [{[R5.2]}] The user shall be able to add edges
        \item [{[R5.3]}] The user shall be able to edit nodes
        \item [{[R5.4]}] The user shall be able to scale nodes
    \end{itemize}
    \item[{[R6]}] The user shall be able to rotate a \gls{city}
    \begin{itemize}
        \item[{[R6.1]}] The user shall be able to rotate a \gls{city} in 45° steps
        \item[{[R6.2]}] The user shall be able to rotate single objects
        \item[{[R6.3]}] The user shall be able to rotate around a center of selected nodes
        \item[{[R6.4]}] The user shall be able to undo the rotation
    \end{itemize}
    \item[{[R7]}] The user shall be able to move a \gls{city}
    \begin{itemize}
        \item[{[R7.1]}] The user shall be able to move single object of a \gls{city}
        \item[{[R7.2]}] The user shall be able to restore the \gls{city} initial position
        \item[{[R7.3]}] The user shall be able to move a \gls{city} or single node in predefined directions
    \end{itemize}
    \item[{[R8]}] The user shall be able to undo and redo actions
    \item[{[R9]}] The user shall be able to lock the camera to a selected \gls{city}
\end{itemize}
\section{Implementation}
\label{section:implementation}
\subsection{Mobile Player}
In this section the mobile player \gls{prefab} will be discussed. 
\gls{see} is supported on multiple platforms and with each platform having different requirements each platform needs to be divided.
Therefore, each platform gets its own player \gls{prefab}.
The mobile player prefab for the \gls{android} version of \gls{see} can be seen in figure \ref{fig:prefab}.

\begin{figure}[htb]
    \centering
    \includegraphics[width=0.8\textwidth]{Implementation/img/mobile_player.png}
    \caption{The mobile player \gls{prefab}}\label{fig:prefab}
\end{figure}

\subsubsection{PlayerMenu Script}
\subsubsection{Prefab interactions}
\subsection{Player Movement}
\subsection{Mobile Menu}
\begin{figure}[htb]
    \centering
    \includegraphics[width=1\textwidth]{Implementation/img/quickmenu.png}
    \caption{The \textit{quickmenu} on the left top side of the mobile device. Pressing the button with an orange marked arrow will expand the menu.}\label{fig:quickmenu}
\end{figure}

\begin{figure}[htb]
    \centering
    \includegraphics[width=1\textwidth]{Implementation/img/menu.png}
    \caption{The player interaction menu on the right top side of the mobile device. The button on the top right side indicate the active interaction mode. Pressing the same button also expands the menu.}\label{fig:menu}
\end{figure}
\section{Evaluation}
\label{section:evaluation}
In the following chapter the mobile implication of \enquote{\gls{see}} will be evaluated in a user study.
Therefore, the mobile application will be compared with the desktop version.

This chapter will start with a description of the desktop version and its main differences in section \ref{desktop}. 
Continuing with a defined aim and the precise hypotheses for the user study in section \ref{aim}.
After sketching the first experiment set up in section \ref{experiment} the actual experiment set up will be discussed in detail in section \ref{real} including the used survey tool, questionnaires and the pilot study.
\subsection{"SEE" Desktop}
\label{desktop}
In this section the desktop version of \enquote{\gls{see}} will be explained. 
In this evaluation the mobile version of \enquote{\gls{see}} will be compared with the desktop version.
Therefore, it is necessary to take a deeper look at the differences between those two versions.
Especially at how the interactions differ and what impact it could have on the user experience.

One outstanding difference from the desktop version to the mobile version is the selection of the interaction modes.
While in the mobile version the menu for the interaction modes is always visible in the desktop version by pressing space a menu screen opens as seen in figure \ref{fig:menu}.
Alternatively interaction modes can be changed by pressing one of the "1-9" keys, which, however requires the user to memorize which number belongs to which mode.

\begin{figure}[htb]
  \centering
  \includegraphics[width=0.8\textwidth]{Evaluation/img/menu.png}
  \caption{The desktop menu for selecting interaction modes.}\label{fig:menu}
\end{figure}

Another difference is the type of user input.
The desktop version uses mouse hovering to display the name of a hovered \enquote{\gls{node}} or \enquote{\gls{plane}}.
This is a faster method then touching the object first in the mobile version. 
In addition to that in the mobile version the object also has to be deselected otherwise there will be a lot of \enquote{\gls{node}} and \enquote{\gls{plane}} names displayed, and it will soon get quite messy.
Also, the precision of object selection differs because touch input can never be as precise as selecting with a mouse cursor. 
This could force the mobile user to zoom further in because with a touch input it will not be possible to select small objects like it might be with a cursor.
Which, of course, would require more time.

One more key difference is the available keyboard for desktop users.
It allows using \enquote{\glspl{shortcut}}, which makes some menu items unnecessary but also requires the user to memorize those \enquote{\glspl{shortcut}}.
The desktop version for example uses the "R" key in the move and rotation mode to recenter or rerotate a \enquote{\gls{city}}.
In the mobile version on the other side the user will find a button for both actions. 
With the right amount of training both actions should probably equal in the amount of time they need but the mobile version sacrifices screen space for those buttons.
If however the user has to type more text like in renaming objects, the common desktop keyboard should come in handy as a study from \cite{kim2014differences} shows that even at a same keyboard size, a virtual one will lack in productivity.
\subsection{Aim and hypothesis}
\label{aim}
The aim of this user study is to answer the research question discussed in section \ref{research}.
In order of answering the research question the finished prototype of the mobile extension shall be evaluated. 
Therefore, the system shall be compared on Android smartphones as well as desktop computers. 
Comparing these two use cases shall give insight on how much impact the constraints of mobile devices have on the usability and overall user experience.
To measure the difference between the desktop and the mobile version the following hypotheses will be used.
The two aspects performance and usability will be measured in the following study and each aspect will have a null hypothesis and an alternative hypothesis.
\begin{enumerate}[{label=\alph*)}]
  \item \textbf{Performance:} The time needed for task in \enquote{\gls{see}} desktop will be called $t_D$ and for mobile $t_M"$.
  \begin{itemize}
    \item \textit{Null Hypothesis $H_{a0}$:}
    \item \textit{Alternative Hypothesis $H_{a1}$:}
  \end{itemize}
  \item \textbf{Usability:} Two aspects are measured for \enquote{\gls{usability}}. First the \enquote{\gls{ASQ}} 
  \begin{enumerate}[label=\roman*)]
    \item ASQ
    \begin{itemize}
      \item \textit{Null Hypothesis $H_{a0}$:}
      \item \textit{Alternative Hypothesis $H_{a1}$:}
    \end{itemize}
    \item SUS
    \begin{itemize}
      \item \textit{Null Hypothesis $H_{a0}$:}
      \item \textit{Alternative Hypothesis $H_{a1}$:}
    \end{itemize}
  \end{enumerate}
\end{enumerate}
\subsection{Experiment set up}
\label{experiment}
The system shall be tested in two groups each starting with a different device. 
Each group does the test on both devices, but one group will start with the mobile application and the other one with the desktop application.
The participants will be assigned random to the groups.
The testers will have various tasks to test the usability of the two applications. 
Afterwards the users will get a survey in English to document their impressions.
\begin{figure}[htb]
  \centering
  \includegraphics[width=1\textwidth]{Evaluation/img/city_1.png}
  \caption{The first \enquote{\gls{city}} for the user study}\label{fig:city1}
\end{figure}

\begin{figure}[htb]
  \centering
  \includegraphics[width=1\textwidth]{Evaluation/img/city_2.png}
  \caption{The second \enquote{\gls{city}} for the user study}\label{fig:city2}
\end{figure}

\begin{figure}[htb]
  \centering
  \includegraphics[width=1\textwidth]{Evaluation/img/city_3.png}
  \caption{The third \enquote{\gls{city}} for the user study}\label{fig:city3}
\end{figure}
In this survey the subjects will be asked various demographic questions as well as what Android device and version they will be using.
In addition to that the subjects will be asked if they are experienced with \enquote{\gls{see}} and if they are experienced with software development.
Before the subjects will be asked to solve various tasks they will be asked to watch a short tutorial video on each application.
After the video they will get a training task where every subject can get used to the system and ask questions if they have trouble solving the training task. 
The overseer will also make sure that every essential action will be practiced such as zooming and moving the \enquote{\gls{city}}.
Figure \ref{fig:city1} shows a small arranged \enquote{\gls{city}} that shall be used for the training tasks.
The structure of the training \enquote{\gls{city}} is generic and follows a simple pattern. 
That shall ensure that the user can focus on the training and that the user does not get overwhelmed. 

Following the first questions and the training, the subjects can start with the main tasks.
For each application there will be two tasks and after each task the subjects will be handed a \enquote{\gls{post-task}} questionnaire.
Last but not least there will another questionnaire that aims to scale the \enquote{\gls{usability}} of the two applications.
For the \enquote{\gls{post-task}} questions the \enquote{\gls{ASQ}} will be used and for the \enquote{\gls{usability}} questions \enquote{\gls{sus}} will be used.
Both questionnaires will be discussed later on in section \ref{questionaires}.
For each main task the overseer will also take the completion time of every main task.
The first and second task on the first device will be performed on the \enquote{\gls{city}} that can be seen in figure \ref{fig:city2} and the third and forth task on device two will be performed on the \enquote{\gls{city}} that can be seen on figure \ref{fig:city3}.
These examples are much larger than the training \enquote{\gls{city}} and represent real life code.
The second \enquote{\gls{city}} shows the file system of Linux and the third one shows the network component of Linux. 
That way the tasks might reflect better on real world uses for \enquote{\gls{see}}.

To not exhaust the testers too much the experiment shall not take longer than one hour. 
This also ensures that there is no to little variance due to exhaustion.
Each participant might have a different concentration span, but this shall not be the focus of this experiment. 

\subsection{Realization}
\label{real}
\subsubsection{Survey tool}
\subsubsection{Questionnaires}
\label{questionaires}
\subsubsection{Pilot study}
In a first test the pilot study was executed with one tester. 
Afterwards the study was discussed and checked for errors. 
It stood out that the example \enquote{\gls{city}} of task one was too different to the one in the second task.
Therefore, the \enquote{\gls{city}} of the first task was exchanged with a larger and better comparable one.
Further on a \enquote{\gls{city}} with 1288 nodes (see figure \ref{fig:city2}) as well as one with 1464 nodes (see figure \ref{fig:city3}) will be used.

Also, the tasks were not comparable because they differed in the types of interactions they used.
In one task the user was asked to rename a node and in the other one the user shall add four nodes.
For renaming a node the user has to use a keyboard which does not make it comparable to just click and add nodes in the second task.

\begin{figure}[htb]
    \centering
    \includegraphics[width=1\textwidth]{Evaluation/img/task1.png}
    \caption{The two key nodes are marked with a yellow arrow}\label{fig:task1}
\end{figure}

\subsubsection{Final experiment set up}

Demographic questions:
\begin{itemize}
    \item Age
    \begin{itemize}
        \item 0-15 years old
        \item 16-30 years old
        \item 31-45 years old
        \item 46+ years old
    \end{itemize}
    \item What gender do you identify as?
    \begin{itemize}
        \item Male
        \item Female
        \item Other ...
        \item Prefer not to say
    \end{itemize}
    \item What is the highest degree or level of education you have completed?
    \begin{itemize}
        \item Some High School (Hauptschule/Realschule...)
        \item High School (Abitur)
        \item Bachelor's Degree
        \item Master's Degree
        \item Ph.D. or higher
        \item Prefer not to say
        \item Other ...
    \end{itemize}
\end{itemize}

Questions regarding used hardware and experience
\begin{itemize}
    \item Are you already experienced with See? 
    \item Do or did you play first person video games?
    \item Do or did you develop software? 
    \item On which Android device will you attend?
    \item Which Android version are you using?*
\end{itemize}

\begin{table}[]
    \resizebox{\textwidth}{!}{%
    \begin{tabular}{lll}
    Nr. &
      Task &
      Expected time \\ \hline
    Training &
      \begin{tabular}[c]{@{}l@{}}Navigate through the planes "dir\_root" \\ \textgreater "dir\_B" \textgreater "dir\_B\_2". On that plane \\ select "b2\_b.cpp" and rename it "b42".\end{tabular} &
      1 - 5 mins \\
    1 &
      \begin{tabular}[c]{@{}l@{}}Detect the largest plane "xfs". On that\\ plane find plane "scrub". Then find \\ and delete node "alloc.c".\end{tabular} &
      0.5 - 5 mins \\
    2 &
      \begin{tabular}[c]{@{}l@{}}Find the plane with one blue child \\ plane ("btrfs"). On the blue child \\ plane "tests" add four new nodes.\end{tabular} &
      1 - 5 mins \\ \hline
    Training &
      \begin{tabular}[c]{@{}l@{}}Navigate through the planes "dir\_root" \\ \textgreater "dir\_C" \textgreater "dir\_C\_2". On that plane \\ select "c2\_b.cpp" and rename it "c42".\end{tabular} &
      1 - 5 mins \\
    3 &
      \begin{tabular}[c]{@{}l@{}}Detect the \\ largest plane "netfilter". On that plane\\ find plane "ipset". Then find and \\ delete node "pfxlen.c".\end{tabular} &
      0.5 - 5 mins \\
    4 &
      \begin{tabular}[c]{@{}l@{}}On the plane with the most \\ edges ("ipv6") find the smallest plane\\ "ila" and connect all four nodes on it.\end{tabular} &
      1 - 5 mins \\ \hline
    \end{tabular}%
    }
    \caption{The tasks used for the experiment. The device will be switched after task 2.}
    \label{table:tasks}
    \end{table}
    
\begin{table}[]
    \resizebox{\textwidth}{!}{%
    \begin{tabular}{llll}
    Phase          &      & \multicolumn{2}{l}{Description}                     \\ \hline
    Pre-Experiment &      & \multicolumn{2}{l}{Demographic questionnaire}       \\ \hline
                   & City & Group 1                  & Group 2                  \\ \hline
    Training       & Figure \ref{fig:city1}   & \multirow{6}{*}{Desktop} & \multirow{6}{*}{Mobile}  \\
    Task 1         & Figure \ref{fig:city2}    &                          &                          \\
    ASQ            &      &                          &                          \\
    Task 2         & Figure \ref{fig:city2}    &                          &                          \\
    ASQ            &      &                          &                          \\
    SUS            &      &                          &                          \\ \hline
    Training       & Figure \ref{fig:city1}   & \multirow{6}{*}{Mobile}  & \multirow{6}{*}{Desktop} \\
    Task 3         & Figure \ref{fig:city3}   &                          &                          \\
    ASQ            &      &                          &                          \\
    Task 4         & Figure \ref{fig:city3}    &                          &                          \\
    ASQ            &      &                          &                          \\
    SUS            &      &                          &                         
    \end{tabular}%
    }
    \caption{Experimental procedure per subject. The procedure is swapped per group.}
    \label{table:procedure}
    \end{table}


\subsubsection{Execution}
\section{Conclusion}
\label{section:conclusion}
Further on this thesis will be closed with a conclusion.
Therefore, the research question: \enquote{\textit{Are Android smartphones suitable of working with \gls{see}?}} will be in focus one last time.
The research question can be interpreted into several sub questions. 
\begin{itemize}
    \item Is it possible to use \gls{see} on \gls{android} devices?
    \begin{itemize}
        \item Yes it is possible with the given implementation from chapter \ref{section:implementation}.
        The given implementation however is only a first approach and does not cover every functionality from the desktop version.
        Functions like reviewing source code are not implemented yet and need further solutions. 
    \end{itemize}
    \item Is it possible to run \gls{see} on any given \gls{android} device?
    \begin{itemize}
        \item No it is not as the results of chapter \ref{section:evaluation} show. 
        Two of the 20 participants could not attend the user study because the mobile version of \gls{see} crashed on their devices.
        18 of 20 participants however were able to use \gls{android}. 
        Five of these 18 participants reported performance issues.
        The bad performance of a device could have caused a worse result as discussed in section \ref{sec:experinece}.
        In summary, that means, in the current state of \gls{see}, an \gls{android} device with solid hardware is needed for a carefree use.
    \end{itemize}
    \item Does it make sense to use the mobile version or is the \gls{usability} too low? 
    \begin{itemize}
        \item The results discussed in section \ref{sec:usability} show that the \gls{sus}-score of the mobile version is significantly lower than the one of the desktop version.
        However, this does not have to mean that mobile version is not useful. 
        \cite{doi:10.1080/10447310802205776} provided a scale to rate the value of a \gls{sus}-score (see figure \ref{fig:sus_scale}).
        Accordingly, to Banger at al. a \gls{sus}-score below 50 means almost certainly that the product will have issues regarding \gls{usability}.
        On the other side a score between 70 and 90 is promising that the product will have a high acceptability.
        The \gls{sus}-score of the mobile version is therefore in the acceptable range even though it is significantly lower than the desktop versions score.
        It can probably be said that the mobile version is useful to have even though there is potential for improvement.
    \end{itemize}
\end{itemize}

\begin{figure}[htb]
    \centering
    \includegraphics[width=1\textwidth]{Conclusion/img/sus.png}
    \caption{A comparison of mean System Usability Scale (SUS) scores by quartile,
    adjective ratings, and the acceptability of the overall SUS score by \cite{doi:10.1080/10447310802205776}}\label{fig:sus_scale}
  \end{figure}

  All in all the research question can be answered with \enquote{yes}.
  Regardless of the facts that the subjects needed significantly less time to solve the tasks in the user study with the desktop, the mobile version still seems to be acceptable and therefore suitable to be used.
  This can be additionally substantiated with the results from the \gls{ASQ} questionnaires, which just slightly differed between the two version.
  Also, as mentioned earlier, the \gls{sus}-score, even if significantly lower, seems to be in an acceptable range.
  It was also to expect that the outcome of a mobile version can hardly be better than the desktop version because of all the constraints a small device brings.
  The named answer of the research question is, of course, only a suggestion, since questions like these cannot be answered completely objectively.

\subsection{Further Study and Ideas}
The presented version of \gls{see} for \gls{android} devices is just a first approach and far from perfect. 
In the following ideas of improvement or further studies will be discussed.

\begin{itemize}
    \item The first thing that comes to mind is the performance of the application on certain devices. 
    Working on a small device often requires a zooming interaction.
    Zooming gets harder if the application does not run smoothly. 
    Future work could focus on improving this performance for a better user experience or at least find out what is the minimum requirement to an \gls{android} device.
    \item Some subjects mentioned concerns regarding the interface. 
    A few devices have camera notches on their display that are in the way of the menu and for other devices the joysticks were too far in the corners, so that they could not be handled properly.
    It is also thinkable that the menu could be improved in general with better icons and color.
    Improvements like these would require further user studies. 
    \item As mentioned in section \ref{sec:restructure} it was intended to restructure \gls{see} in order to minimize the need of conditional compilation.
    This process is far from trivial and would require further work. 
    \item The current version has only been tested for \gls{android} smartphones.
    It would be interesting how \gls{see} performs on tabled devices and on other operating systems like iOS.
    \item Another aspect to look at could be \gls{ar}. \cite{santos2016guidelines} provided guidelines for a mobile \gls{ar} interface, which could be adapted to \gls{see}. 
    Adding \gls{ar} to \gls{see} could have great potential and offer new innovative ways of cooperating. 
    It would also be interesting to compare the existing \gls{vr} version of \gls{see} with an \gls{ar} approach.
\end{itemize}

\subsection{Closing Words}

This thesis has started with describing the motivation and research question in chapter \ref{sec:introduction}.
To answer this question fundamental information was given in chapter \ref{sec:fundamentals} before a concept for the mobile version of \gls{see} was sketched in chapter \ref{section:concept}.
The concept was then realized almost exactly as planned in chapter \ref{section:implementation}.
It was also discussed were the implementation faced problems and how they were solved.
Then, in chapter \ref{section:evaluation}, the evaluation was planed, executed and analyzed. 
Therefore, a significance level of $\alpha = 0.05$ was used.
20 subjects participated even though only 18 could finish the user study. 
The user study gave insight on aspects like \textit{performance}, \textit{usability}, \textit{complexity} and many more.
Finally, in chapter \ref{section:conclusion} the research question was answered with \enquote{yes}, even though the answer to this question can never be completely objective and the desktop version of \gls{see} did perform better in general.

\appendix

\ifdef{\glossaryFile}{
    % Print all glossaries here
    \printglossaries
}{}

% List of Figures.
\listoffigures

\vspace{8ex}

% List of Tables. Uncomment if you want to include.
\listoftables

\vspace{8ex}

\Regie{Kontrolliere am Ende, ob alle bibliographischen Angaben
  vollständig sind.  Wird also die Zeitschrift oder Konferenz
  aufgeführt, in der ein Artikel veröffentlicht wurde?  Sind überall
  die Seitenangabe aufgeführt? Bei Verweisen auf Web-Seiten, ist
  überall angegeben, wann der letzte Zugriff darauf erfolgte? Sind
  Umlaute und andere Sonderzeichen korrekt in LaTeX beschrieben
  worden?}

\section{Related Files}

\begin{table}[H]
    \resizebox{\textwidth}{!}{%
    \begin{tabular}{ll}
    \hline
    \textbf{Name}    & \textbf{Description} \\ \hline
    SEE\_Desktop.zip\label{file:desktop} &                      \\
    SEE\_Mobile.apk  &                      \\
    calc\_data.ipynb\label{calc} & \begin{tabular}[c]{@{}l@{}}A Python script that calculates all \\ the results of the study.\end{tabular}
    \end{tabular}%
    }
\end{table}


% Finally, the bibliography.
\ifdef{\sourcesFile}{
\bibliographystyle{unsrtnat}
\bibliography{\sourcesFile}
%\printbibliography[heading=bibnumbered]
}{}
\addcontentsline{toc}{section}{References}
\end{document}
