\section{Einführung}

\Regie{Diese Vorlage ist aufgeteilt in die Kapitel Einführung,
  Grundlagen und verwandte Arbeiten, Konzept, Umsetzung, Evaluation,
  Zusammenfassung und Ausblick. Das ist eine generische, klassische
  Aufteilung. In Einzelfällen kann auch eine andere Aufteilung
  vorgenommen werden, aber letztlich muss zu all diesen Punkten etwas
  in der Arbeit gesagt werden. Auch die Titel der Kapitel können
  eventuell angepasst werden.} 

\Regie{Ein allgemeiner Hinweis: Eine Abschlussarbeit als
  wissenschaftliche Arbeit hat immer zwei Ebenen: Auf der konkreten
  Ebene wird transparent dargestellt, was gemacht wurde. Auf der \glqq
  Metaebene\grqq{} wird reflektiert, warum man das und nicht etwas
  Anderes gemacht hat. Hierzu werden also Alternativen explizit
  genannt und gegeneinander abgewogen. Es wird begründet, warum man
  sich dann für das konkrete Vorgehen entschieden hat.}
  
\Regie{In der Einführung wird die Aufgabenstellung präzise formuliert
  und geeignet motiviert.  Danach folgt eine kurze Übersicht über den
  Rest des Dokuments.}

\Regie{Bitte gleich zum Punkt kommen. Bei der Motivation nicht erst
  langatmig ausholen, wie komplex und allgegenwärtig Software ist. Das
  wissen die Gutachter(innen) und haben es schon viel zu oft nochmals
  lesen müssen.}

\Stil{Zwischen einer Überschrift und seines ersten Unterabschnitts
  muss ein kurzer Text kommen. Zum Beispiel kann man hierin
  beschreiben, was den Leser in diesem Kapitel erwartet.}

\Stil{Ganz am Ende, wenn das Dokument fertig zu sein scheint und man
  es eigentlich nur noch ausdrucken möchte, dann möge man bitte zuvor
  noch einmal das Inhaltsverzeichnis ansehen. Dort ist zu prüfen, ob
  bei einer Unterteilung eines Abschnittes auch immer mindestens zwei
  solche Unterteilungen existieren. Immer wieder kommt es vor, dass es
  beispielsweise einen Unterabschnitt 2.3.1, aber keinen
  Unterabschnitt 2.3.2 gibt.  Wenn es nur einen Unterabschnitt gibt,
  dann ergibt eine Unterteilung keinen Sinn.}

\Stil{Bitte auf die Rechtschreibung achten. Viele Arbeiten enthalten
  viel zu viele Fehler. Die Schreibweise von Worten kann man
  nachschlagen. Es gibt Tools, die einfache Tippfehler erkennen. Es
  gibt Regeln für Kommata. Rechtschreibung kann man lernen.  Es gibt
  keine Ausrede für mangelhafte Rechtschreibung, wenn man nicht an
  Legasthenie leidet.  Natürlich werden immer einige Fehler schlicht
  übersehen werden, wenn es aber zu viele davon gibt, stört das den
  Lesefluss und macht den schlechten Eindruck mangelnder Sorgfalt.
  Man sollte immer eine andere Person, die sicher in der
  Rechtschreibung ist, bitten, das Dokument durchzulesen, weil man
  sich seiner eigenen Fehlern vielleicht nicht bewusst ist.  Es eignet
  sich dafür eine Person besonders gut, die vom Inhalt selbst gar
  nicht viel versteht und sich allein auf die Form konzentrieren
  kann.}

\Stil{Fußnoten sollten möglichst vermieden werden, weil sie den Leser
  zwingen, den Kontext zu wechseln. Will man einen Artikel
  referenzieren, dann nimmt man ohnehin eine Zitierung; siehe
  unten. Ansonsten verwendet man Klammern oder Gedankenstriche, um den
  Text direkt in den Lesefluss einzubetten. Und vielleicht ist das,
  was man in die Fußnote packen möchte, ohnehin irrelevant, dann kann
  man es auch ganz weglassen. Wenn man etwas in eine Fußnote steckt,
  muss man sowieso damit rechnen, dass es keiner liest.}
  
\section{Grundlagen und verwandte Arbeiten}

\Regie{Hier werden alle wesentlichen und nur die wesentlichen
  Grundlagen, die nicht als bekannt vorausgesetzt werden können,
  aufgeführt. Manchmal liest man Arbeiten, die den Mangel an eigenem
  Beitrag mit einem überbordenden Grundlagenkapitel zu kompensieren
  versuchen. Bei allem, was hier beschrieben wird, stets kritisch
  reflektieren, ob erstens davon auszugehen ist, dass den
  Gutachter(innen) das bereits nicht schon zur Genüge bekannt sein
  dürfte, und zweitens, ob das in einem späteren Kapitel auch wirklich
  wieder aufgegriffen wird.}

\Regie{Bachelor- und Masterarbeiten sind wissenschaftliche
  Abschlussarbeiten. Demzufolge gehört es auch dazu, dass man sich mit
  der wissenschaftlichen Literatur auseinandergesetzt hat. Man sollte
  dabei keineswegs nur die Literatur einfach nur wiedergeben, sondern
  sich auch kritisch mit ihr auseinandersetzen. Insbesondere ist
  darzulegen, wie Aspekte der gesichteten Literatur Einfluss auf die
  eigene Arbeit genommen haben. Zur Frage, wie man Literatur
  systematisch sucht und zusammenfasst, gibt es in der Literatur eine
  Reihe von Empfehlungen \citep{Kitchenham:TR:07, Brereton:JSS:07,
    Petersen:EASE:08, Kitchenham:IST:09, Wohlin:JSS:13,
    Kitchenham:EASE:10, Petersen:IST:15}. Insbesondere der technische
  Bericht von \cite{Kitchenham:TR:07} stellt eine ausführliche
  Anleitung dar.}

\Stil{Wobei wir beim Zitierstil wären. Für Zitierungen gibt es im
  Deutschen tatsächlich eine Norm: die DIN 1505. Hier werden die
  Autoren -- zumindest partiell -- und die Jahreszahl für die
  Zitierung verwendet. Das ist zwar länglicher als reine Nummern oder
  kryptische Abkürzungen, wie sie gerne in Aufsätzen mit wenig Platz
  benutzt werden, enthält aber mehr lesbare und erinnerbare
  Informationen. Außerdem kann man sie besser in den Satz einbauen,
  wie es oben schon beim Verweis auf den Bericht von
  \cite{Kitchenham:TR:07} gezeigt wurde.}

% Das Folgende ist nur zur Demonstrieren.
% Lösch ruhig erstmal alles aus dieser Datei, sobald du mit der Arbeit anfängst.

\if\disableChapter1
\subsection{Glossarbeispiel}
\ifdef{\glossaryFile}{ Es können auch Glossarbegriffe verwendet werden
  (falls Glossar.tex so konfiguriert wurde), im Kontext von \gls{see}
  könnte man \zB{} über \glspl{city} reden wollen.
  \if\defineOnFirstUse1 Du siehst, wie die neuen Begriffe als
  \texttt{\detokenize\expandafter{\firstUseCommand}}angezeigt werden.
  \else Du hast kein \verb+\firstUseCommand+ festgelegt, deswegen
  werden die Begriffe ohne ihre Definition abgedruckt (die ist
  natürlich aber trotzdem im Glossar).
  \fi%
  Für mathematische Glossarbegriffe würde man das so machen:
  $\glssymbol{S}_{\emptyset} = \{\emptyset\}$. \firstUseSymbol{S}

  Nachdem man jetzt den Begriff \enquote{\gls{see}} oder
  \enquote{\gls{city}} schonmal verwendet hat, wird er auch ganz
  normal angezeigt (ist aber immer noch anklickbar).  Wenn die
  Glossarbegriffe irgendwie komisch sind oder kein Glossar auf den
  folgenden Seiten erscheint, führe auf der Kommandozeile
  \texttt{makeglossaries Arbeit} aus.}{Du hast kein Glossar
  definiert, was aber auch nicht schlimm ist.}

\subsection{Abbildungsbeispiel}
\begin{figure}[htb]
    \centering
    \includegraphics[width=0.7\textwidth]{unibremen}
    \caption{Logo der Universität Bremen.}\label{fig:uni}
\end{figure}

Als Beispiel ist hier in \autoref{fig:uni} das Logo der Uni Bremen.
Wenn du ein etwas breiteres Bild hast, das sich bis in die Ränder
erstrecken soll, verwende statt \verb+\begin{figure}+ einfach
\verb+\begin{figure*}+. Im Moment wird die Abbildung unten
abgedruckt, verwende ansonsten wie üblich
\verb+\begin{figure}[htbp]+ um auch andere Platzierungen zu
erlauben.

Wenn du die Position eines Bilds forcieren willst, würde ich von dem
oft verwendeten Ansatz \verb+\begin{figure}[H]+ abraten, da das manche
Layouts zerschießen kann.  Verwende stattdessen lieber Umgebungen,
die keine Floats sind\footnote{ Siehe bspw.\
\url{https://tex.stackexchange.com/a/8631}.  }, \zB{}
\texttt{minipage} oder einfach \texttt{center}.

\ifdef{\sourcesFile}{ Zitate mit Bib\LaTeX{} funktionieren wie immer,
  als Beispiel hier ein Zitat zum sog.\
  Hawthorne-Effekt~\citep{Hawthorne1939}.  Die Referenzen kommen
  einfach in die \texttt{\sourcesFile}.  }{Du hast scheinbar keine
  Bibliographie eingerichtet, was für eine wissenschaftliche Arbeit
  ungewöhnlich ist. Das solltest du tun. } \else Wenn du das hier
lesen kannst sieht diese Beispielseite vermutlich ziemlich hässlich
aus, weil ich überall \verb+\section+ statt \verb+\chapter+ verwendet
habe.  Ich hab deswegen erstmal die meisten Beispiele entfernt, um es
nicht schlimmer zu machen. Du kannst dir natürlich immer noch einfach
den Quellcode ansehen.  \fi

\vspace{1em}

{\large \emph{Dann viel Erfolg beim Schreiben der Arbeit%
\ifthenelse{\equal{Dein Name\xspace}{\myName}}{}{, {\saveexpandmode\expandarg\StrBefore{\myName}{ }\restoreexpandmode}}!}}

\section{Konzept}

\Regie{Hierin werden die Anforderungen und die Ziele der Arbeit
  nochmals präzisiert und motiviert und Entwurfsentscheidungen
  formuliert, wie diese Ziele durch diese Arbeit erreicht
  werden. Abhängig vom Gegenstand der Arbeit sollten hier auch
  explizit die möglichen Nutzergruppen, dereren Charakteristika und
  Ziele aufgeführt werden. Das letztlich gewählte Konzept sollte nicht
  einfach nur wiedergegeben werden, vielmehr sind auch mögliche
  Alternativen zu nennen und abzuwägen. Es soll nachvollziehbar
  begründet dargelegt werden, warum man sich für genau jenes Konzept
  entschieden hat.}

\section{Umsetzung}

\Regie{Hier wird die Umsetzung (oder auch: Implementierung) näher
  ausgeführt. Sie stellt einen Übergang vom Konzept zum
  implementierten Code dar. Der Code selbst wird im elektronischen
  Anhang oder in Form eines Branches in einem für die Gutachter(innen)
  zugänglichen Repository abgegeben. Die Details des Codes selbst
  gehören somit nicht in die Ausarbeitung. Vielmehr wird die
  Implementierung abstrakter beschrieben. Wenn es nicht-triviale
  Algorithmen zu beschreiben gibt, kann das mit Hilfe von Pseudo-Code
  getan werden. Es können auch UML-Diagramme verwendet werden; diese
  sollten dann bitte auch in korrekter UML-Notation angegeben
  werden. Häufig findet man Klassendiagramme, die einen Überblick über
  die Implementierung geben sollen. Leider sind diese oft so
  umfangreich, dass ihre Abbildung herunterskaliert wird, so dass die
  Texte in den Grafiken nicht mehr lesbar sind. Oft finden sich darin
  Details, auf die im Text gar nicht eingegangen wird, wie (private)
  Attribute und Methoden. Ein Klassendiagramm kann durchaus sinnvoll
  sein, um den Zusammenhang zu erläutern, aber letztlich ist es nur
  ein Bild, das im Text auch erklärt werden muss. Wird etwas davon im
  Text nicht erklärt, ist es entweder irrelevant und kann weggelassen
  werden oder der Text ist nicht vollständig.}

\Regie{Wenn auf anderen Arbeiten aufgebaut wurde, dann muss der eigene
Beitrag deutlich davon abgegrenzt werden. Was wurde übernommen? Was
verändert? Was wurde neu hinzugefügt?}

\Regie{Um das, was man entwickelt hat, besser darzustellen, kann man
  auch noch ein Video drehen. Das bietet sich insbesondere für
  interaktive, graphische Dinge an. Der oder die Zweitgutachter(in)
  wird kaum selbst das entwickelte Programm bauen und ausführen. Für
  diese Person ist es insbesondere hilfreich, wenn sie sich das in
  einem Video vorführen lassen kann.}

\section{Evaluation}

\Regie{Die adäquate Form der Evaluation hängt vom Gegenstand ab,
  insofern wird sich die Darstellung der Evaluation in diesem Kapitel
  von Arbeit zu Arbeit unterscheiden. Allen gemein ist jedoch, dass es
  sich um eine empirische Studie handelt. Egal, ob nur die Laufzeit
  eines Algorithmus gemessen wird oder ob ein kontrolliertes
  Experiment durchgeführt wurde. Bei
  \url{https://seafile.zfn.uni-bremen.de/published/swt/} findet man
  Folien und Videos zur empirischen Softwaretechnik. Dort erfährt man
  viel Nützliches, wie man eine Studie aufbaut und evaluiert.}

\Regie{Die Evaluation muss transparent und nachvollziehbar dargestellt
  werden. Die Ausarbeitung sollte alles so detailliert wiedergeben,
  dass eine andere Person die Studie replizieren kann. Dabei kann für
  sehr detaillierte Aspekte auch auf den Anhang zurückgegriffen
  werden. Dann sollte im Text in diesem Kapitel dennoch eine kurze
  Zusammenfassung dieser Aspekte und der explizite Verweis auf den
  Anhang mit den Details gemacht werden.}
  
\subsection{Forschungsfragen}

\Regie{Hier werden die leitenden Forschungsfragen, die in der
  Evaluation untersucht werden sollen, präzise
  beschrieben. Idealerweise in Form operationalisierter Hypothesen.
  Es wird erläutert, warum diese Forschungsfragen relevant sind.}

\subsection{Versuchsaufbau}

\Regie{Hier wird erklärt, wie die Studie gestaltet ist. Auch wird
  dargelegt, warum der Versuchsaufbau den Forschungsfragen angemessen
  ist, welche Alternativen es gegeben hätte und warum diese verworfen
  wurden. Insbesondere werden hier möglichst vollständig die
  unabhängigen Variablen, die einen Einfluss auf das Versuchsergebnis
  haben können, aufgezählt. Es wird beschrieben, welche davon wie in
  der Studie variiert werden, wenn es sich um ein kontrolliertes
  Experiment handelt. Dann werden vollständig alle abhängigen
  Variablen beschrieben. Sie ergeben sich aus den operationalisierten
  Hypothesen. Die Art und Weise wird präzise beschrieben. Kommen
  hierfür Fragebögen zum Einsatz, wird erläutert, wie der Fragebogen
  zusammengestellt wurde und warum die Fragen relevant sind. Für viele
  Fragestellungen -- insbesondere zur Usability -- gibt es bereits
  Standardfragebögen in der Literatur. Dann sollte man auch auf diese
  zurückgreifen, oder aber überzeugend darlegen, warum diese nicht
  passen. Gibt es mehrere solche Fragebögen, dann werden sie alle
  vorgestellt und gegeneinander abgewogen und einer begründet
  ausgewählt (oder möglicherweise auch mehrere kombiniert).}

\Regie{Der Versuchsaufbau sollte mindestens einmal in einer
  Pilotstudie ausprobiert werden. Damit erzielt man Erkenntnisse, wie
  lange die Probanden später pro Durchlauf brauchen werden und ob sie
  alle gestellten Aufgaben auch tatsächlich verstehen. Die Ergebnisse
  der Pilotstudie können nicht für die Auswertung später verwendet
  werden, weil die Situation eine andere ist und vermutlich sich durch
  die Erfahrungen der Pilotstudie auch noch Änderungen im
  Versuchsaufbau ergeben. Die Erkenntnisse der Pilotstudie und die
  daraus resultierten Änderungen am Versuchsaufbau werden explizit
  genannt.}

\subsection{Probanden und Objekte der Studie}

\Regie{Nehmen an einer Studie Menschen teil, spricht man von ihnen in
  der Regel als Probanden. Nicht sie selbst sind der Gegenstand der
  Untersuchung; vielmehr werden sie benötigt, um etwas untersuchen zu
  können, weil es nicht durch einen Algorithmus automatisiert werden
  kann. Objekte einer Studie sind die Dinge, an denen etwas untersucht
  wird. Das können zum Beispiel Programme sein, in denen die Probanden
  Fehler suchen müssen. Probanden und Objekte sind letztlich auch
  unabhängige Variablen. Das Ergebnis hängt sehr von ihrer Auswahl
  ab. Insofern müssen sie sorgfältig ausgewählt werden, damit die
  Ergebnisse der Studie aussagekräftig und verallgemeinerbar sind. In
  den oben bereits angeführten Videos und Folien zur empirischen
  Softwaretechnik findet man verschiedene Sampling-Methoden, anhand
  derer Probanden und Objekte ausgewählt werden können. Der Einsatz
  systematischer Sampling-Methoden kommt bei den Objekten eher in
  Frage als bei den Probanden. In der Regel kommt realistischerweise
  bei den Probanden nur das Convenience-Sampling zum Einsatz, weil es
  schwierig ist, überhaupt Probanden zu finden, und man in der Regel
  auch nichts über die Gesamtpopulation der Software-Entwicklerinnen
  und -Entwickler weiß. Das ist ein \textit{Threat to Validity} und
  muss offen im Abschnitt~\ref{sec:threats} aufgeführt werden. Um zu
  untersuchen, ob es tatsächlich einen Zusammenhang zwischen
  Eigenschaften der Probanden (\zB{} spezielle Vorkenntnisse mit dem
  Untersuchungsgegenstand oder Programmiererfahrung) gibt, kann man
  auch statistische Untersuchungen zu Korrelation dieser Eigenschaften
  mit den abhängigen Variablen anstellen. In jedem Falle müssen
  relevante Charakteristika sowohl der Probanden (\zB{} Alter,
  Geschlecht, Programmiererfahrung etc.) als auch der Objekte (\zB{}
  Größe und Herkunft der in der Untersuchung verwendeten Programme)
  aufgeführt werden. }

\Regie{Bei der Untersuchung sind auch ethische Fragen und Belange des
  Datenschutzes der Probanden zu berücksichtigen. Dazu sollten vorher
  alle Probanden aufgeklärt worden sein und ein Einverständnis zum
  Datenschutz unterzeichnet haben.}

\subsection{Gestellte Aufgaben}

\Regie{Hier werden die Aufgaben präzise erläutert, die die Probanden
  zu erledigen hatten. Es muss dargelegt werden, warum diese Aufgaben
  repräsentativ für die untersuchten Fragestellungen sind. In der
  Regel sind die Aufgabenbeschreibungen kurz genug, dass man sie hier
  im Text verbatim wiedergeben kann. Sollte das nicht der Fall sein,
  dann kann man sie im Anhang im Detail aufführen und sie hier kurz
  zusammenfassen.}

\subsection{Ergebnisse}

\Regie{Hier werden die quantitativen Ergebnisse wiedergegeben
  (zusammenfassend \zB{} in Form von Boxplots, aber auch in Rohform,
  \zB{} in Form von Tabellen oder Scatterplots) und mit Hilfe passender
  statistischer Test auf signifikante Unterschiede geprüft. Neben der
  Prüfung der Hypothesen untersucht man auch, ob es möglicherweise
  Korrelationen von Eigenschaften der Probanden und Objekte mit den
  abhängigen Variablen gibt.}

\subsection{Diskussion}

\Regie{Hier werden die Ergebnisse in Beziehung zueinander gesetzt und
  interpretiert. Dazu gehören auch qualitative Erkenntnisse, die nach
  Erklärungen für das Beobachtete suchen. Hierzu schaut man sich die
  Daten genauer an, nicht nur anhand statistischer Parameter. Bei der
  Diskussion ist explizit zu machen, was durch Daten wirklich gestützt
  ist und an welchen Stellen man den Boden der Spekulation betritt.}

\subsection{Threats to Validity}
\label{sec:threats}

\Regie{Hier werden die internen und externen \emph{Threats to
    Validity} erläutert, wie das zu jeder empirischen Studie
  gehört. Welche davon wurden im Vorfeld schon bedacht und wie das
  Design entsprechend angepasst? Welche existieren weiterhin?}

\section{Zusammenfassung und Ausblick}

\Regie{Hier werden die wesentlichen Ergebnisse der Arbeit
  zusammengefasst und Schlussfolgerungen aus den Erkenntnissen gezogen
  und mögliche Verbesserungen oder Anschlussfragen diskutiert. Eine
  kritische Selbstreflexion über das eigene Vorgehen, den Prozess und
  das Erreichte runden das Ganze ab.}

\Regie{Das Dokument soll in duplex gedruckt werden. Es muss gebunden
  sein. Ein inhaltsgleiches PDF wird den beiden Gutachtern geschickt
  und auch im elektronischen Anhang mit abgegeben.}

\appendix{Anhang}

\Regie{Es gibt einen elektronischen Anhang für den abgegebenen Code,
  umfangreichere Messdaten oder auch Demo-Videos. Weniger umfangreiche
  Ergänzungen wie \zB{} die Fragebögen, kleinere Mengen von Messdaten
  etc.\ können in diesem Anhang hier in Papierform mitabgegeben
  werden.}

\Regie{Der elektronische Anhang wird in der Regel als USB-Stick
  abgegeben. CDs sind auch möglich, wenn auch mittlerweile eher
  weniger gebräuchlich. In jedem Falle sollte es dort eine
  README-Datei geben, die die Ordnerstruktur erläutert und auch den
  Namen des Autors und den Titel der Arbeit nennt.}
