% Template by Falko Galperin <falko1@uni-bremen.de>, 2021

%%% USAGE INFO HERE %%%

% Add any acronyms or glossary terms you wish to define to the list at the 
% bottom of this file, the format is specified there.
% To reference glossary terms in your text, use:
% \gls{key} normally,
% \glspl{key} to write the plural form of the glossary term,
% \Gls{key} (and equivalents) to capitalize the glossary term,
% \gls*{key} (and equivalents) to not write the definition on the first use,
% \glssymbol{key} to refer to mathematical symbols (see the bottom of this file!), and
% \Glsdesc{key} to write the description instead of the glossary term.
% You also need to generate the glossary using e.g. `makeglossaries Arbeit`
% in the shell. The template can be configured in the section below.

%%% CONFIGURATION HERE %%%

% Whether a glossary term should be defined (e.g. in the margins) on its first use.
% 1 is true, 0 is false.
\def\defineOnFirstUse{1}

% Which acronym style to use.
% `long-postshort-user` will cause it to be written in its long form the first time,
% and use its short form thereafter.
\def\acronymStyle{long-postshort-user}

% Which color to use for coloring glossary terms on first use.
\def\firstColor{Blue}

% Which color to use for coloring glossary terms on subsequent usages.
\def\secondColor{black}

% What command to use on first use. 
% Set \footnote to use footnotes instead of margin notes.
\def\firstUseCommand{\marginpar}

%%% CONFIGURATION ENDS %%%

\usepackage[toc,numberedsection=autolabel,section=section,abbreviations]{glossaries-extra}
\usepackage{glossary-longextra}
\if\defineOnFirstUse1
\usepackage{marginfix}
\fi

% No links for description
\let\oldglsdesc\glsdesc
\renewcommand{\glsdesc}[1]{\oldglsdesc*{#1}}
\let\Oldglsdesc\Glsdesc
\renewcommand{\Glsdesc}[1]{\Oldglsdesc*{#1}}

\makeglossaries{}

\setabbreviationstyle[acronym]{\acronymStyle}

% Color acronyms short form on first use, too.
\renewcommand{\glsfirstabbrvuserfont}[1]{\textcolor{\firstColor}{#1}}

% Only color on first time...
\newcommand{\firstuseformat}[1]{\textcolor{\firstColor}{{\emph{#1}}}}
% ...not on the second time, third time, and so on.
\newcommand{\seconduseformat}[1]{\textcolor{\secondColor}{{#1}}}

\if\defineOnFirstUse1
\newcommand{\mainformat}[1]{%
    \ifglsused{#1}{}{\glslinkvar{%
        \firstUseCommand{\textcolor{\firstColor}{\emph{\glsentryname{#1}}:} \glsentrydesc{#1}}}{}{}%
    }%
}
\fi


\defglsentryfmt[main]{\glsgenentryfmt\mainformat{\glslabel}}
%\defglsentryfmt[\acronymtype]{\glsgenentryfmt\ifglsused{\glslabel}{\glsentryshort{\glslabel}}{\glsentrylong{\glslabel}}\mainformat{\glslabel}}
\defglsentryfmt[\acronymtype]{\glsgenentryfmt\ifglsused{\glslabel}{}{\mainformat{\glslabel}}}

\renewcommand{\glslinkpresetkeys}{%
  \ifglsused{\glslabel}%
    {\let\glstextformat\seconduseformat}%
    {\let\glstextformat\firstuseformat}%
}

% Special command for symbols.
\newcommand{\glsmarginpar}[1]{
    \firstUseCommand{%
    \textcolor{\firstColor}{\emph{\glsentryname{#1} (\glsentrysymbol{#1})}:} \glsentrydesc{#1}}%
}

\newcommand\firstUseSymbol[1]{\glsunset{#1}\glsmarginpar{#1}}

%%% ACRONYMS HERE %%%

% Format: \newacronym[description={your description}]{yourKey}{yourShortName}{yourLongName}
% `yourKey` will be used to access the glossary term later on, e.g. using \gls{yourKey}.

\newacronym[description={Eine interaktive Visualisierung von Software, welche die \emph{Code-City}-Metapher verwendet und einen kollaborativen Multiplayer über verschiedene Plattformen\footnote{Neben Desktop- und Touch-Umgebungen noch Virtual Reality (\zB{} \emph{Valve Index}) und Augmented Reality (\zB{} \emph{Microsoft HoloLens})} hinweg ermöglicht.},sort=see]{see}{\textsc{see}}{Software Engineering Experience}
\newacronym[description={A post-task questionnaire consisting of three questions that is used to access how difficult a user perceived a task (See \enquote{\gls{post-task}}).}]{ASQ}{ASQ}{After-Scenario Questionnaire}
\newacronym[description={The Sytem Usability Scale consists of ten questions that measure \enquote{\gls{usability}}}]{sus}{SUS}{System Usability Scale}
%%% GLOSSARY HERE %%%

% Format: \newglossaryentry{yourKey}{name=yourSingularName, description={your Description}, plural=yourPlural}
% Note that specifying the plural is optional, by default it will simply be SingularForm + s.
% Later on, you can access the plural using \glspl{key}.

\newglossaryentry{city}{name=Code-City,
description={In der Code-City-Metapher werden Softwarekomponenten durch Gebäude in einer Stadt repräsentiert, wobei die Eigenschaften dieser Gebäude verschiedene Metriken der Software ausdrücken können --- \zB{} könnte die Höhe eines Gebäudes der Anzahl der Codezeilen entsprechen.}, plural=Code-Cities}
\newglossaryentry{post-task}{name={Post-Task},description={A questionnaire that is taken after every task of an experiment.}}
\newglossaryentry{usability}{name={Usability},description={A term that describes how well a (software) system can be used.}}
\newglossaryentry{node}{name={Node},description={A point in a diagram where lines intersect. In \enquote{\gls{see}} it usually displays a software class}, plural={nodes}}
\newglossaryentry{plane}{name={Plane},description={An area that bundeles \enquote{\glspl{node}}. Could for example represent a namespace.}}
\newglossaryentry{edge}{name={Edge},description={A line that connects two \enquote{\glspl{node}}. Could for example represent a class call.}}
\newglossaryentry{shortcut}{name={Shortcut},description={A combination of key that will call an action like for example "ctrl" + "c" for coping a text.}}
%%% MATHEMATICAL SYMBOLS %%%

% Format: \newglossaryentry{yourKey}{name={your written name for the symbol}, category=symbol, description={your description}, symbol={your symbol}}
% Later on, use \glssymbol{yourKey} to use the symbol itself.
% IMPORTANT NOTE: Symbols will not be written to the margins automatically on first use.
% You need to do this yourself by calling \firstUseSymbol{yourKey} OUTSIDE
% OF A MATH (or similar) ENVIRONMENT.

\newglossaryentry{S}{name={Symmetrische Gruppe}, category=symbol,
    description={Die Gruppe aller Permutationen einer Menge $X$, wobei diese Menge in unserer Notation als Index geschrieben wird, also \zB{} $\mathfrak{S}_X$.}, symbol={\ensuremath{\mathfrak{S}}}}
