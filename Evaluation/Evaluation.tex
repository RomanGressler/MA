\section{Evaluation}

\subsection{"SEE" Desktop}
In this section the desktop version of \enquote{\gls{see}} will be explained. 
In this evaluation the mobile version of \enquote{\gls{see}} will be compared with the desktop version.
Therefore, it is necessary to take a deeper look at the differences between those two versions.

\subsection{Aim and hypotheses}
The finished prototype of the mobile extension shall be evaluated. 
Therefor the system shall be compared on smartphones as well as desktop computers. 
Comparing these two use cases shall give insight on how much impact the constraints of mobile devices have on the usability and overall user experience.

The hypotheses of this thesis is that the mobile version of \enquote{\gls{see}} lacks slightly in usability compared to the desktop version.
This due to the constraints of the mobile device. 
A smartphone has far less screen space and also does not have a physical keyboard, which would allow many more shortcuts.
\subsection{Experiment set up}
The system shall be tested in two groups each starting with a different device. 
Each group does the test on both devices, but one group will start with the mobile application and the other one with the desktop application.
The participants will be assigned random to the groups.
The testers will have various tasks to test parameters such as understandability, learnability, operability and attractiveness. 
Afterwards the users will get a survey in English or German to document their impressions.
The survey shall also conclude differences between screen sizes and Android versions. 

To not exhaust the testers too much the experiment shall not take longer than one hour. 
This also ensures that there is not to little variance due to exhaustion.
Each participant might have a different concentration span, but this shall not be the focus of this experiment. 

\begin{figure}[htb]
    \centering
    \includegraphics[width=1\textwidth]{Evaluation/img/city_1.png}
    \caption{The first \enquote{\gls{city}} for the user study}\label{fig:city1}
\end{figure}

\begin{figure}[htb]
    \centering
    \includegraphics[width=1\textwidth]{Evaluation/img/city_2.png}
    \caption{The second \enquote{\gls{city}} for the user study}\label{fig:city2}
\end{figure}

\subsection{Realization}
\subsubsection{Survey tool}
\subsubsection{Pilot study}
In a first test the pilot study was executed with one tester. 
Afterwards the study was discussed and checked for errors. 
It stood out that the example \enquote{\gls{city}} of task one was too different to the one in the second task.
Therefore, the \enquote{\gls{city}} of the first task was exchanged with a larger and better comparable one.
Further on a \enquote{\gls{city}} with 1288 nodes (see figure \ref{fig:city2}) as well as one with 1464 nodes (see figure \ref{fig:city3}) will be used.

Also, the tasks were not comparable because they differed in the types of interactions they used.
In one task the user was asked to rename a node and in the other one the user shall add four nodes.
For renaming a node the user has to use a keyboard which does not make it comparable to just click and add nodes in the second task.


\begin{figure}[htb]
    \centering
    \includegraphics[width=1\textwidth]{Evaluation/img/city_3.png}
    \caption{The third \enquote{\gls{city}} for the user study}\label{fig:city3}
\end{figure}

\begin{figure}[htb]
    \centering
    \includegraphics[width=1\textwidth]{Evaluation/img/task1.png}
    \caption{The two key nodes are marked with a yellow arrow}\label{fig:task1}
\end{figure}

\subsubsection{Final experiment set up}

Demographic questions:
\begin{itemize}
    \item Age
    \begin{itemize}
        \item 0-15 years old
        \item 16-30 years old
        \item 31-45 years old
        \item 46+ years old
    \end{itemize}
    \item What gender do you identify as?
    \begin{itemize}
        \item Male
        \item Female
        \item Other ...
        \item Prefer not to say
    \end{itemize}
    \item What is the highest degree or level of education you have completed?
    \begin{itemize}
        \item Some High School (Hauptschule/Realschule...)
        \item High School (Abitur)
        \item Bachelor's Degree
        \item Master's Degree
        \item Ph.D. or higher
        \item Prefer not to say
        \item Other ...
    \end{itemize}
\end{itemize}

Questions regarding used hardware and experience
\begin{itemize}
    \item Are you already experienced with See? 
    \item Do or did you play first person video games?
    \item Do or did you develop software? 
    \item On which Android device will you attend?
    \item Which Android version are you using?*
\end{itemize}

\begin{table}[]
    \resizebox{\textwidth}{!}{%
    \begin{tabular}{lll}
    Nr. &
      Task &
      Expected time \\ \hline
    Training &
      \begin{tabular}[c]{@{}l@{}}Navigate through the planes "dir\_root" \\ \textgreater "dir\_B" \textgreater "dir\_B\_2". On that plane \\ select "b2\_b.cpp" and rename it "b42".\end{tabular} &
      1 - 5 mins \\
    1 &
      \begin{tabular}[c]{@{}l@{}}Detect the largest plane "xfs". On that\\ plane find plane "scrub". Then find \\ and delete node "alloc.c".\end{tabular} &
      0.5 - 5 mins \\
    2 &
      \begin{tabular}[c]{@{}l@{}}Find the plane with one blue child \\ plane ("btrfs"). On the blue child \\ plane "tests" add four new nodes.\end{tabular} &
      1 - 5 mins \\ \hline
    Training &
      \begin{tabular}[c]{@{}l@{}}Navigate through the planes "dir\_root" \\ \textgreater "dir\_C" \textgreater "dir\_C\_2". On that plane \\ select "c2\_b.cpp" and rename it "c42".\end{tabular} &
      1 - 5 mins \\
    3 &
      \begin{tabular}[c]{@{}l@{}}On the right Code-City detect the \\ largest plane "netfilter". On that plane\\ find plane "ipset". Then find and \\ delete node "pfxlen.c".\end{tabular} &
      0.5 - 5 mins \\
    4 &
      \begin{tabular}[c]{@{}l@{}}On the plane with the most \\ edges ("ipv6") find the smallest plane\\ "ila" and connect all four nodes on it.\end{tabular} &
      1 - 5 mins \\ \hline
    \end{tabular}%
    }
    \caption{The tasks used for the experiment. The device will be switched after task 2.}
    \label{table:tasks}
    \end{table}
    
\begin{table}[]
    \resizebox{\textwidth}{!}{%
    \begin{tabular}{llll}
    Phase          &      & \multicolumn{2}{l}{Description}                     \\ \hline
    Pre-Experiment &      & \multicolumn{2}{l}{Demographic questionnaire}       \\ \hline
                   & City & Group 1                  & Group 2                  \\ \hline
    Training       & Figure \ref{fig:city1}   & \multirow{6}{*}{Desktop} & \multirow{6}{*}{Mobile}  \\
    Task 1         & Figure \ref{fig:city2}    &                          &                          \\
    ASQ            &      &                          &                          \\
    Task 2         & Figure \ref{fig:city2}    &                          &                          \\
    ASQ            &      &                          &                          \\
    SUS            &      &                          &                          \\ \hline
    Training       & Figure \ref{fig:city1}   & \multirow{6}{*}{Mobile}  & \multirow{6}{*}{Desktop} \\
    Task 3         & Figure \ref{fig:city3}   &                          &                          \\
    ASQ            &      &                          &                          \\
    Task 4         & Figure \ref{fig:city3}    &                          &                          \\
    ASQ            &      &                          &                          \\
    SUS            &      &                          &                         
    \end{tabular}%
    }
    \caption{Experimental procedure per subject. The procedure is swapped per group.}
    \label{table:procedure}
    \end{table}


\subsubsection{Execution}