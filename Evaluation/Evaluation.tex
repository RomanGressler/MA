\section{Evaluation}

\subsection{"SEE" Desktop}
In this section the desktop version of \enquote{\gls{see}} will be explained. 
In this evaluation the mobile version of \enquote{\gls{see}} will be compared with the desktop version.
Therefore, it is necessary to take a deeper look at the differences between those two versions.

\subsection{Aim and hypotheses}
The finished prototype of the mobile extension shall be evaluated. 
Therefor the system shall be compared on smartphones as well as desktop computers. 
Comparing these two use cases shall give insight on how much impact the constraints of mobile devices have on the usability and overall user experience.
\subsection{Experiment set up}
The system shall be tested in two groups each starting with a different device. 
Each group does the test on both devices, but one group will start with the mobile application and the other one with the desktop application.
The participants will be assigned random to the groups.
The testers will have various tasks to test parameters such as understandability, learnability, operability and attractiveness. 
While solving these tasks the testers will be asked to think out loud. 
Afterwards the users will get a survey in English or German to document their impressions.
The survey shall also conclude differences between screen sizes and Android versions. 

To not exhaust the testers too much the experiment shall not take longer than one hour. 
This also ensures that there is not to little variance due to exhaustion.
Each participant might have a different concentration span, but this shall not be the focus of this experiment. 